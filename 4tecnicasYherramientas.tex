En este apartado se indicar�n las t�cnicas y herramientas utilizadas durante la realizaci�n del proyecto.



\section{T�cnicas de desarrollo}

En esta secci�n\dots

\section{Herramientas utilizadas}

En este apartado se mostrar� las distintas herramientas utilizadas para el desarrollo del proyecto.

\subsection{Gestor de Tareas: VersionOne}
Se ha estudiado entre varias posibles herramientas, entre ellas est�n:
\begin{itemize}
    \item \href{http://www.pivotaltracker.com/}{PivotalTracker}
    \item \href{https://www.fogcreek.com/FogBugz/}{FogBugz}
    \item \href{http://www.versionone.com/}{VersionOne}
\end{itemize}
\paragraph{}Se ha optado por la herramienta VersionOne, que ofrece unas condiciones notablemente mejores a las otras en su versi�n gratuita y adem�s resulta bastante intuitiva y f�cil de usar.

\subsection{Gestor de Versiones: Git Hub}

Se ha estudiado entre varias posibles herramientas, entre ellas est�n:
\begin{itemize}
    \item \href{https://github.com/}{GitHub}
    \item \href{https://bitbucket.org/}{Bitbucket}
    \item \href{http://sourceforge.net/}{Sourcefroge}
\end{itemize}
\paragraph{}Finalmente se decidi� que se iba a usar la herramienta GitHub porque se ten�a experiencia previa en el uso de la misma, ofrece unas condiciones bastante razonables en su versi�n gratuita y se puede hace un buen seguimiento del proyecto con ella.
\subsection{IDEL de desarrollo: Android Studio}
Se ha estudiado entre varias posibles herramientas, entre ellas est�n:
\begin{itemize}
    \item \href{https://eclipse.org/}{Eclipse}
    \item \href{http://developer.android.com/sdk/index.html}{Android Studio}
\end{itemize}
\paragraph{}La elecci�n de Android Studio ha sido porque no s�lo es una herramienta exclusivamente dedicada a aplicaciones Android, sino que resultaba m�s prometedora que Eclipse; la cu�l pensamos que puede quedar obsoleta para este tipo de aplicaciones.

\subsection{Herramienta de Generaci�n de Documentaci�n: \LaTeX}
Se ha elegido esta herramienta debido a su facilidad de uso y a que optimiza autom�ticamente la estructura del producto final para ofrecer el mejor resultado visual.

\subsection{Herramientas de Deep Learning: Deep Belief SDK}

Se ha estudiado entre varias posibles herramientas, entre ellas est�n:
\begin{itemize}
    \item \href{http://caffe.berkeleyvision.org/}{Caffe}
    \item \href{http://libccv.org/post/with-a-sub-10-image-classifier-a-decent-face-detector-here-comes-ccv-0.7/}{Lib CCV}
    \item \href{http://cs.stanford.edu/people/karpathy/rcnn/}{Overfeat}
    \item \href{https://github.com/jetpacapp/DeepBeliefSDK}{Deep Belief SDK}
\end{itemize}
\paragraph{}La herramienta que se ha escogido aportaba grandes ventajas debido a que viene con ejemplos para utilizarla en diferentes plataformas y su uso era relativamente f�cil, mucho m�s f�cil en comparaci�n al resto de herramientas.

\subsection{Manual del Programador}
En esta secci�n se proceder� a la explicaci�n detallada de c�mo instalar las herramientas necesarias y qu� herramientas son necesarias para trabajar sobre este proyecto.
\subsubsection{Instalaci�n del JDK}
La primera, y m�s esencial de las herramientas, es el JDK de java, que es el set o conjunto de herramientas y librer�as para los desarolladores de java.

Primero deberemos ir a la p�gina de \href{http://cs.stanford.edu/people/karpathy/rcnn/}{Oracle} en la que descargaremos el jdk, la p�gina deber�a tener un aspecto m�s o menos como este
\figura{1}{imgs/jdk_1.png}{P�gina de descarga del JDK de Java}{figImagen1}{}\\
En dicha p�gina tendremos que aceptar la licencia y posteriormente descargar el JDK que sirva para nuestra m�quina. Una vez hemos descargado dicho archivo, lo ejecutamos. Una vez ejecutado seguimos los siguientes pasos para su instalaci�n.
\figura{1}{imgs/jdk_2.png}{Instalaci�n del JDK, paso 1.}{figImagen2}{}\\
\figura{1}{imgs/jdk_3.png}{Instalaci�n del JDK, paso 2.}{figImagen3}{}
\figura{1}{imgs/jdk_4.png}{Instalaci�n del JDK, paso 3.}{figImagen4}{}\\
\figura{1}{imgs/jdk_5.png}{Instalaci�n del JDK, paso 4.}{figImagen5}{}


