\label{chap:Introduccion}
Este proyecto tiene como objetivo el estudio te�rico de herramientas de \textit{machine learning}. Las herramientas que se quieren estudiar tendr�n como funcionalidad el tratamiento de im�genes y la extracci�n de una predicci�n a partir de estas.

Como un segundo objetivo, se pretende crear una peque�a aplicaci�n de prueba. Que usar� una o varias herramientas de las estudiadas para mostrar el funcionamiento de las mismas.

La aplicaci�n de soporte para los usuarios ser� desarrollada en Android y constar� de una interfaz y funcionamiento orientada al uso por personas con dificultades de visi�n, adem�s de guiar� al usuario a trav�s de la aplicaci�n con un asistente de voz que ir� diciendo al usuario qu� debe hacer y en qu� punto de la aplicaci�n se encuentra.

La aplicaci�n tendr� como objetivo inicial mandar una foto tomada por el usuario a un servidor que la procesar� y extraer� una predicci�n que le dir� al usuario lo que se puede observar en la imagen que �l ha elegido. La predicci�n ser� una frase que describa la imagen y est� ser� le�da por la aplicaci�n.

Por otra parte en el lado del servidor usaremos algoritmos de \textit{machine learning} para el procesado de la imagen y la extracci�n de la predicci�n. El servidor recibir� una petici�n de tipo POST y cargar� la imagen, la cu�l ser� procesada y en respuesta devolver� una frase en espa�ol que describa la imagen.

El principal objetivo del proyecto es conseguir crear un prototipo de esta aplicaci�n que tenga un correcto funcionamiento y que sirva como base para desarrollar sobre �l una aplicaci�n m�s optimizada y de a�n mejor funcionamiento.