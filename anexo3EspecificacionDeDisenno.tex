\section{Introducci�n}
En este apartado se proceder� a explicar las especificaciones de dise�o que se han ido utilizando para el desarrollo de la aplicaci�n.


\section{Dise�o en el Servidor}
En primer lugar se introducir� las especificaciones que se han seguido para desarrollar el servidor.
\subsection{API RESTful}
Se ha construido el servidor de manera que siga, en la medida de lo posible, las especificaciones de la API RESTful. Esto se puede observar en el hecho de que se ha usado Flask para programar el servidor, que es un \textit{framework} que permite la programaci�n de peticiones de la API RESTful de manera muy sencilla. Adem�s, tiene soporte para el uso de urls a la hora de trabajar con peticiones y funciones.

En el servidor se ha trabajado s�lo con una url, la url principal del servidor. En la imagen (\ref{lst:URL}) se puede observar c�mo se ha definido la url base para el servidor. Adem�s podemos ver que esta soporta dos tipos de m�todos, o bien la petici�n de tipo GET, o bien la petici�n de tipo POST.
\begin{lstlisting}[frame=none,language=Python,caption={Url �nica y principal del servidor},basicstyle=\large,label={lst:URL}]
@app.route("/", methods=['GET', 'POST'])
def index():
\end{lstlisting}
Si nos fijamos, tenemos una funci�n justo debajo de la notaci�n ``app.route'', esto implica que al acceder a esa url, se ejecuta esta funci�n y es ah� donde se determina qu� hace el servidor en cada caso.

En el servidor, la funci�n trata de manera predeterminada la petici�n como una petici�n de tipo GET, as� que habr� que tratar el m�todo POST de alguna manera (\ref{lst:POST}).
\begin{lstlisting}[frame=none,language=Python,caption={Tratando el m�todo POST},basicstyle=\large,label={lst:POST}]
if request.method == 'POST':
	#Aqu� lo que se har�a si es m�todo POST
	return "la respuesta de la operaci�n"
#Aqu� lo que se har�a si es m�todo GET
return "respuesta de un m�todo GET"
\end{lstlisting}
EL objeto ``request'' es un objeto que usa Flask internamente y representa la petici�n que se est� recibiendo. Para acceder a qu� tipo de petici�n es, accedemos a su variable ``method'' en la que obtendremos la respuesta y con este valor se trabaja la petici�n POST.

Para asegurar la seguridad del sistema se ha usado una lista (\ref{lst:TIPOS}) de tipo de archivos admisibles por el servidor, lo que impide que se nos mande archivos maliciosos con extensiones extra�as, pues si el archivo tiene una extensi�n incorrecta este descarta la petici�n.
\begin{lstlisting}[frame=none,language=Python,caption={Tipos admitidos por el servidor},basicstyle=\large,label={lst:TIPOS}]
ALLOWED_EXTENSIONS = set(['jpg','JPG','jpeg','JPEG'])
\end{lstlisting}
Posteriormente se define una funci�n que se encarga de comprobar si el tipo es uno de los permitidos y si se cumple la condici�n, entonces se contin�a el proceso, de lo contrario se descarta el archivo recibido.
\begin{lstlisting}[frame=none,language=Python,caption={Comprobando tipos},basicstyle=\large,label={lst:FLAG}]
def allowed_file(filename):
    return '.' in filename and \
           filename.rsplit('.', 1)[1] in ALLOWED_EXTENSIONS
\end{lstlisting}
