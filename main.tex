%
% Incluir el estilo de la Universidad de Deusto.

\documentclass{memoriaPFC}
%Para evitar un error por falta de registros..
\usepackage{etex}
\usepackage{microtype}




%
% Incluir el estilo definido para la Universidad de Burgos por �lvar Arn�iz 2009/12/20.
\include{estiloUBU}

%
% Incluir los comandos para las palabras t�cnicas.
\include{comandosIS}

%
% DATOS DEL DOCUMENTO
\title{Aplicaci�n de soporte a personas con dificultades de visi�n}

%\autores{Nombre1 Apellidos1}{Nombre2 Apellidos2}
\autores{Bryan Reinoso Cevallos}{}

\tutor{Jos� Francisco D�ez Pastor, \\\textcolor{white}{Tutores}Dr. C�sar I. Garc�a Osorio}


\date{marzo de 2015}

\titulacion{ii}

\resumen{%
}
\descriptores{}

\abstract{%
}
\keywords{}


%
% COMIENZO DEL DOCUMENTO
\begin{document}

%
% Portada, resumen, indices
\frontmatter
\ubuportada
\hacerresumen
\tableofcontents
\listoffigures % Opcional
\listoftables % Opcional
%\lstlistoflistings % Opcional

%
% Contenidos
\mainmatter
%\subsection*{Agradecimientos}
%Quiero aprovechar este peque�o espacio\dots
%
\chapter{Introducci�n}
\label{chap:Introduccion}
El objetivo principal del\dots
%
\chapter{Objetivos del proyecto}
\section{Objetivos funcionales}
El principal objetivo del proyecto es tener un prototipo de la aplicaci�n que funcione de manera adecuada,para esto tenemos que tener en cuenta tanto los objetivos del lado del servidor como del lado del cliente.
\subsection{Objetivos funcionales en el cliente Android}
El principal objetivo que podemos encontrarnos en el lado del cliente es el disponer de un prototipo de aplicaci�n Android con la que el usuario pueda mandar de manera intuitiva una imagen al servidor y recibir de �l la predicci�n esperada. Para que esto funcione de manera adecuada se han propuesto una serie de requisitos que deben cumplirse:
\begin{itemize}
    \item Para poder enviar la imagen al servidor y recibir de �l la predicci�n, se debe establecer una conexi�n con el desde la aplicaci�n Android y hacer un correcta petici�n de tipo POST. Para ello usamos la librer�a de Apache con el m�dulo MultipartEntityBuilder, que nos permite establecer de manera bastante sencilla la conexi�n y mandar la petici�n POST. Adem�s por especficaciones de Android esto deber� realizarse en un hilo separado del hilo principal.
    \item Para que la persona que lo utilice no tenga por qu� saber la estructura de la interfaz gr�fica de la aplicaci�n para poder usarla sin problemas, nos apoyaremos sobre los eventos ontouch de Android y evitaremos lo m�ximo posible que el usuario tenga que saber c�mo es la interfaz.
    \item Para ofrecer una gu�a f�cil y �til para el usuario a lo largo de su experiencia usando la aplicaci�n, usaremos la librer�a Text2Speech para ir guiando al usuario con instrucciones en voz alta explic�ndole lo que debe hacer en cada momento. Finalmente le devolveremos la predicci�n y se le dir� cu�les son sus opciones.
    \item Para poder tomar la imagen se requerir� de un dispositivo con c�mara de fotos y se tendr� que establecer los permisos sobre la aplicaci�n para el uso de la misma. 
\end{itemize}
\subsection{Objetivos funcionales en el servidor Flask}
En el lado del servidor tenemos un objetivo claro y es el de recibir una imagen a trav�s de una petici�n POST desde un cliente, procesarla y, finalmente, devolver la predicci�n en espa�ol. Para conseguir todo esto tenemos una serie de requisitos y objetivos que tenemos que cumplir:
\begin{itemize}
	\item Para poder recibir las peticiones y mandar respuestas tenemos que tener un servidor con la capacidad de gestionar este tipo de operaciones, para ello se ha decidido usar Flask para programar el servidor. Con Flask se crear� un servidor que gestione la petici�n post sobre un URL concreto .
	\item Para poder procesar la imagen se necesita una red neuronal a trav�s de la cual pasar la imagen y extraer la predicci�n. Para ello se ha elegido un proyecto desarrollado por \href{https://github.com/karpathy}{Karpathy} y su proyecto \href{https://github.com/karpathy/neuraltalk}{Neuraltalk}. Para ello se instalar� el proyecto en la m�quina servidora y todas las dependencias del mismo. Adem�s se har� todas las modificaciones necesarias para su correcto funcionamiento sobre nuestra aplicaci�n.
	\item Una vez la predicci�n haya sido extra�da esta viene en ingl�s, as� que el servidor se deber� encargar de traducirla al espa�ol para enviarla al cliente. Para ello se usa una conexi�n con el traductor de Google. 
\end{itemize}
%
\chapter{Conceptos te�ricos}
En este apartado se profundizar�\dots
%
\chapter{T�cnicas y herramientas}
En este apartado se indicar�n las t�cnicas y herramientas utilizadas durante la realizaci�n del proyecto.



\section{T�cnicas de desarrollo}

En esta secci�n\dots

\section{Herramientas utilizadas}

En este apartado se mostrar� las distintas herramientas utilizadas para el desarrollo del proyecto.

\subsection{Gestor de Tareas: VersionOne}
Se ha estudiado entre varias posibles herramientas, entre ellas est�n:
\begin{itemize}
    \item \href{http://www.pivotaltracker.com/}{PivotalTracker}
    \item \href{https://www.fogcreek.com/FogBugz/}{FogBugz}
    \item \href{http://www.versionone.com/}{VersionOne}
\end{itemize}
\paragraph{}Se ha optado por la herramienta VersionOne, que ofrece unas condiciones notablemente mejores a las otras en su versi�n gratuita y adem�s resulta bastante intuitiva y f�cil de usar.

\subsection{Gestor de Versiones: Git Hub}

Se ha estudiado entre varias posibles herramientas, entre ellas est�n:
\begin{itemize}
    \item \href{https://github.com/}{GitHub}
    \item \href{https://bitbucket.org/}{Bitbucket}
    \item \href{http://sourceforge.net/}{Sourcefroge}
\end{itemize}
\paragraph{}Finalmente se decidi� que se iba a usar la herramienta GitHub porque se ten�a experiencia previa en el uso de la misma, ofrece unas condiciones bastante razonables en su versi�n gratuita y se puede hace un buen seguimiento del proyecto con ella.
\subsection{IDEL de desarrollo: Android Studio}
Se ha estudiado entre varias posibles herramientas, entre ellas est�n:
\begin{itemize}
    \item \href{https://eclipse.org/}{Eclipse}
    \item \href{http://developer.android.com/sdk/index.html}{Android Studio}
\end{itemize}
\paragraph{}La elecci�n de Android Studio ha sido porque no s�lo es una herramienta exclusivamente dedicada a aplicaciones Android, sino que resultaba m�s prometedora que Eclipse; la cu�l pensamos que puede quedar obsoleta para este tipo de aplicaciones.

\subsection{Herramienta de Generaci�n de Documentaci�n: \LaTeX}
Se ha elegido esta herramienta debido a su facilidad de uso y a que optimiza autom�ticamente la estructura del producto final para ofrecer el mejor resultado visual.

\subsection{Herramientas de Deep Learning: NeuralTalk}

Se ha estudiado entre varias posibles herramientas, entre ellas est�n:
\begin{itemize}
    \item \href{http://libccv.org/post/with-a-sub-10-image-classifier-a-decent-face-detector-here-comes-ccv-0.7/}{Lib CCV}
    \item \href{http://cs.stanford.edu/people/karpathy/rcnn/}{Overfeat}
    \item \href{https://github.com/jetpacapp/DeepBeliefSDK}{Deep Belief SDK}
\end{itemize}
\paragraph{}Esta herramienta ha sido elegida porque ven�a en el mismo idioma en el que se iba a programar el servidor, adem�s de que su aplicaci�n en nuestro proyecto era mucho mas pr�ctico, pues devuelve una frase descriptiva de lo que en una imagen hay.

\subsection{Herramientas de Deep Learning: Caffe}

Esta herramienta es usada en nuestro proyecto debido a que la herramienta NeuralTalk tiene dependencia de este proyecto para extraer caracter�sticas de las im�genes y luego poder realizar una predicci�n con ellas.
Adem�s este proyecto est� bien estructurado y en si se sigue correctamente su documentaci�n es relativamente f�cil de instalar.

\subsection{Herramientas de programaci�n: MATLAB}

Tuvimos que instalar MATLAB porque NeuralTalk usa un script de MATLAB para poder preparar las im�genes para extraerles las caracter�sticas, adem�s se usa el wrapper de caffe para MATLAB.

\subsection{Herramientas de desarrollo de servidores: Flask}

Se ha estudiado entre varias posibles herramientas, entre ellas est�n:
\begin{itemize}
    \item \href{http://axis.apache.org/axis2/c/core/}{Axis2/C}
    \item \href{http://www.cs.fsu.edu/~engelen/soap.html}{GSOAP}
    \item \href{http://tomcat.apache.org/}{Tomcat}
\end{itemize}
\paragraph{}Se ha escogido esta herramienta en concreto porque sobre todas las dem�s su funcionamiento era muy inmediato y adem�s se escribe en Python, que es un idioma muy vers�til y f�cil de usar. El hecho de que esta herramienta tenga un funcionamiento y programaci�n tan sencilla la hace una herramienta que, a nuestro parecer, destaca sobre el resto y es interesante trabajar con ella.


\subsection{Manual del Programador}
En esta secci�n se proceder� a la explicaci�n detallada de c�mo instalar las herramientas necesarias y qu� herramientas son necesarias para trabajar sobre este proyecto.
\subsubsection{Instalaci�n del JDK}
La primera, y m�s esencial de las herramientas, es el JDK de java, que es el set o conjunto de herramientas y librer�as para los desarolladores de java.

\paragraph{} Primero deberemos ir a la p�gina de \href{http://cs.stanford.edu/people/karpathy/rcnn/}{Oracle} en la que descargaremos el jdk, la p�gina deber�a tener un aspecto m�s o menos como este
\figura{1}{imgs/jdk_1.png}{P�gina de descarga del JDK de Java}{figImagen1}{}\\
En dicha p�gina tendremos que aceptar la licencia y posteriormente descargar el JDK que sirva para nuestra m�quina. Una vez hemos descargado dicho archivo, lo ejecutamos. Una vez ejecutado seguimos los siguientes pasos para su instalaci�n.
\figura{1}{imgs/jdk_2.png}{Instalaci�n del JDK, paso 1.}{figImagen2}{}\\
\figura{1}{imgs/jdk_3.png}{Instalaci�n del JDK, paso 2.}{figImagen3}{}
\figura{1}{imgs/jdk_4.png}{Instalaci�n del JDK, paso 3.}{figImagen4}{}\\
\figura{1}{imgs/jdk_5.png}{Instalaci�n del JDK, paso 4.}{figImagen5}{}



%
\chapter{Estado del arte}
\label{chap:EstadoArte}
En este apartado se procede a la presentaci�n de una serie de art�culos que han conformado el estudio te�rico que conlleva este proyecto. el cu�l tiene un gran peso en el mismo; pues el estudio de cada una de las herramientas que se han tenido en cuenta tiene por detr�s un concienzudo estudio por parte del alumno, el cu�l ha tenido que comprender el funcionamiento de las mismas a trav�s de este estudio.
\section{Art�culo del DeepBelief SDK}
\label{subchap:EstadoDeepBelief}

\section{Art�culo del NeuralTalk}
\label{subchap:EstadoNeuralTalk}

\section{Art�culo del Arctic Caption}
\label{subchap:EstadoArctic}

\section{Art�culo del Google}
\label{subchap:EstadoGoogle}
%
\chapter{Aspectos relevantes del desarrollo del proyecto}
En este apartado se introducen los aspectos m�s relevantes del proyecto.

\section{Dificultades encontradas}
Durante el desarrollo del proyecto nos hemos  encontrado varias dificultades que han hecho que el proyecto se retrase considerablemente y su avance no haya sido ni f�cil ni r�pido.
\subsection{Dificultades con GSOAP y Apache}
Este fallo es el que m�s ha retrasado al proyecto y ha supuesto una dificultad enorme a la hora de llevar a cabo el mismo.
\paragraph{}Empezamos con que para usar \href{http://www.cs.fsu.edu/~engelen/soap.html}{GSOAP} se tuvo que estudiar una serie de cosas para poder adquirir los conocimientos necesarios para usar la herramienta, dichos conocimientos ser�n listados aqu�:
\begin{itemize}
    \item \href{http://axis.apache.org/axis2/c/core/}{XML:} Se tuvo que coger un nivel adecuado en el uso de XML ya que la herramienta GSOAP se basa en el uso de este tipo de archivos como medio de comunicaci�n en las distintas peticiones y respuestas que procesa. Adem�s el XML tambi�n es necesario para comprender el funcionamiento de SOAP y de WSDL, los cuales son completamente necesarios para entender el funcionamiento de la herramienta GSOAP.
    \item \href{http://www.cs.fsu.edu/~engelen/soap.html}{SOAP:} Esta especificaci�n se tuvo que estudiar para comprender el funcionamiento de la herramienta GSOAP y en qu� se basaba su funcionamiento, entender el por qu� deb�a funcionar la herramienta y como se realiza la comunicaci�n gracias a ella. Aunque el SOAP no es usado directamente cuando usas GSOAP es necesario conocer esta especificaci�n ya que GSOAP s� que usa WSDL, para el cual tenemos que tener un conocimiento b�sico, al menos, de SOAP para poder usarlo.
    \item \href{http://tomcat.apache.org/}{WSDL:} Esta otra especificaci�n s� que se usa directamente en la herramienta GSOAP y b�sicamente con ella vertebras toda la aplicaci�n que vas a hacer, de hecho tienes dos opciones:
    \subitem La primera es usar un archivo WSDL donde especificas las operaciones que el  servidor va a realizar, despu�s con la herramienta GSOAP generas todos los stubs y documentos necesarios para hacer tu aplicaci�n.
    \subitem La segunda ser�a a trav�s de un documento de tipo .h o una cabecera de C. Con el cu�l tambi�n generas los stubs y documentos necesarios para programar tu servidor, entre dichos documentos se econtrar� un archivo WSDL que contendr� la especificaci�n de las operaciones que hay dentro del fichero cabecera que hayas usado. Pero incluso en esta opci�n necesitas entender SOAP y WSDL porque a trav�s de comentarios tienes que especificar caracter�sticas que ir�n directamente al fichero WSDL, y que ser�n necesarios para el correcto funcionamiento del servidor.
\end{itemize}
\paragraph{}Una vez se ha estudiado lo anterior se paso al estudio de la documentaci�n de la herramienta GSOAP, adem�s de el intento de hacer que funcionen sus ejemplos.Cuando se consigui� que funcionar�n sus ejemplos se paso a la programaci�n de un servidor propio, una vez se programo y se hicieron las pruebas de que estaba bien programado se procedi� a intentar que este funcionar� desde un cliente GSOAP.
Para que funcionar� con el cliente GSOAP se hizo una investigaci�n de c�mo hacer que el servidor funcionar� en localhost y, siguiendo la recomendaci�n que en la documentaci�n de GSOAP encontramos, se instal� Apache y se inteto usar el m�dulo de Apache para su funcionamiento con GSOAP.
\paragraph{}Como conclusi�n sacamos que, tras una larga investigaci�n y mucho tiempo dedicada a esta herramienta, esta herramienta no tenia la documentaci�n suficiente como para poder hacerla funcionar con Apache y, a pesar de haberlo intentado  muchas veces, no conseguimos que el servidor GSOAP programado por nosotros devolviera alguna vez un resultado coherente al cliente.
Finalmente desechamos la opci�n de trabajar con esta herramienta y le dimos un giro al proyecto con el que esper�bamos tener avances m�s r�pidos y mejores, optamos por la programaci�n de un servidor en Flask.
\subsection{Dificultades con DeepBeliefSDK}
Al final no se pudo usar.
\subsection{Dificultades en la instalaci�n de Herramientas}
Caffe tuvimos problemas de compatibilidad con open-cv
Neuraltalk tuvimos problemas con la cantidad de dependencias que ten�a.
\subsection{Dificultades con NeuralTalk}
Tuvimos problemas con la herramienta de MATLAB al ejecutar un script desde el servidor.
\subsection{Dificultades con Android}
Problemas al conectarnos con el servidor.
%
\chapter{Conclusiones y l�neas de trabajo futuras}
En este apartado se presentar�n las conclusiones y las posibles l�neas de futuro que saquemos del proyecto.
\section{Conclusiones del Proyecto}
\section{Lineas de trabajo Futura}

%
% ANEXOS
% Redefinir el nombre de cap�tulo por Anexo y que sean n�meros romanos
\renewcommand\chaptername{Anexo}
\renewcommand\thechapter{\Roman{chapter}}
\setcounter{chapter}{0}

% A�adir entrada en el �ndice: Anexos
\addcontentsline{toc}{chapter}{Anexos}
%
%\portadasAuxiliares{Anexo I - Plan del proyecto software}
%\chapter{Plan del proyecto software}
%\section{Plan de Proyecto Software}
Este apartado est� dedicado al plan de proyecto software donde se comenta todo el proceso de planificaci�n del proyecto.

\subsection{Introducci�n}
La planificaci�n del proyecto se lleva a cabo de forma �gil, con la metodolog�a SCRUM. Y la plataforma sobre la que se ha trabajado para realizar la planificaci�n es \href{http://www.versionone.com/}{VersionOne}. Y la planificaci�n se ha dividido en \textit{sprints} o iteraciones, las cu�les tienen la mayor�a una duraci�n de una semana, aunque por problemas algunas han durado m�s.

\subsubsection{Problemas encontrados}
Debido a que las primeras dos iteraciones se realizan de forma simulada, osea despu�s del tiempo previsto, por eso no se pueden sacar sus gr�ficos \textit{burn-down}. Adem�s de que al d�a 30 de uso del \href{http://www.versionone.com/}{VersionOne} se caduc� la prueba y perd� todos los datos del proyecto, pero contactando con el soporte de \href{http://www.versionone.com/}{VersionOne} estos me reabrieron la prueba por una semana para extraer los datos necesarios, pero debido a la tardanza en su respuesta hay un \textit{sprint} que dura 3 semanas y  no podemos extraer su gr�fico \textit{burn-down} tampoco.

\subsection{Planificaci�n temporal del proyecto}
En este apartado se presentan los distintos \textit{sprints} y las tareas de cada uno.

\subsubsection{Sprint 1:23 de Diciembre al 12 de Febrero}
Este \textit{sprint} es de una duraci�n bastante mayor al resto porque se empez� con mucha antelaci�n y su objetivo era el de documentarse. Se estudi� los distintos aspectos del proyecto y se tomo contacto con el mismo.

\paragraph{}Este \textit{sprint} es importante porque es donde se asientan las bases del proyecto y se empieza a tomar contacto con el proyecto. Sin este \textit{sprint} se hubiera empezado sin ninguna base y el comienzo del proyecto hubiera sido m�s tard�o y , por tanto, su avance m�s lento.

\subsubsection{Sprint 2:12 de Febrero al 26 de Febrero}
Esta iteraci�n se dedic� exclusivamente a la decisi�n de las herramientas que se van a usar en el proyecto y a la familiarizaci�n con las mismas. 

Aqu� se decidi�:
\begin{itemize}
	\item Gestor de versiones: \href{https://github.com/}{GitHub}
	\item Gestor de Tareas: \href{http://www.versionone.com/}{VersionOne}
	\item Entorno de desarrollo: \href{http://developer.android.com/sdk/index.html}{Android Studio}
	\item Herramienta de ofim�tica: \LaTeX con su editor TeXMaker
\end{itemize}

\subsubsection{Sprint 3:26 de Febrero al 6 de Marzo}
En este \textit{sprint} se instalaron y se crearon cuentas de las herramientas del anterior \textit{sprint}. Se dividi� en distintas tareas:
\begin{itemize}
	\item Crear cuenta \href{http://www.versionone.com/}{VersionOne}
	\item Crear cuenta en \href{https://github.com/}{GitHub}
	\item Instalar \href{http://developer.android.com/sdk/index.html}{Android Studio}
	\item Instalar librer�a DeepBeliefSDK
\end{itemize}
\paragraph{}Adem�s se procedi� a invitar a los tutores a \href{http://www.versionone.com/}{VersionOne}, realizar el primer commit en GitHub, probar ejemplos de Android y del DeepBeliefSDK y, finalmente, se gener� la documentaci�n asociada a este \textit{sprint}.
\figura{1}{imgs/S1_1.png}{Burn-down del sprint 3}{Sprint1}{}\\

Podemos ver el gr�fico \textit{burn-down} de este sprint en la imagen \ref{Sprint1}.

\subsubsection{Sprint 4:6 de Marzo al 13 de Marzo}
En esta iteraci�n se profundiza, sobretodo, en la utilizaci�n de la librer�a DeepBeliefSDK, se probar�n ejemplos y trabajar�n sobre ellos. Adem�s se generar� la documentaci�n asociada al \textit{sprint}.

Se identifican las tareas:
\begin{itemize}
	\item Consulta de la documentaci�n de Android: se pasa a avanzar en la programaci�n de Android y se empiezan a realizar las primeras aplicaciones de prueba con ayuda de la documentaci�n de Android.
	\item Instalaci�n Linux: Se procede a instalar un Linux con su distribuci�n Ubuntu en una m�quina virtual sobre la que trabajar.
	\item Impresiones de Pantalla: Se hacen capturas de pantalla del proceso de instalaci�n de las herramientas para la futura documentaci�n en la secci�n: Manual del programador (\ref{anx:ManualProgramador})
\end{itemize}

\paragraph{}Este fue el primer \textit{sprint} en el que a�adimos el \textit{Retrospective Meeting} y el \textit{Sprint Planning} al \href{http://www.versionone.com/}{VersionOne}, pese a que s� que hab�amos hecho estas reuniones.

\paragraph{}En el \textit{Retrospective Meeting} se determino que:

\begin{itemize}
	\item Se estudi� el funcionamiento de la librer�a DeepBeliefSDK
	\item Se hicieron las impresiones de pantalla para la documentaci�n
	\item Se hicieron m�s pruebas con Android.
	\item Se envi� invitaciones a los profesores a VersionOne y Github
\end{itemize}

\paragraph{}El \textit{Sprint Planning} simplemente sirvi� para determinar las tareas a realizar en esta iteraci�n, las cu�les ya han sido comentadas antes.

\figura{1}{imgs/S2_1.png}{Burn-down del sprint 4}{Sprint2}{}
\paragraph{}Adem�s esta iteraci�n viene acompa�ada con un gr�fico \textit{burn-down}s el cu�l puede verse en la imagen \ref{Sprint2}

\subsection{Sprint 5:13 de Marzo al 27 de Marzo}
Este \textit{sprint} est� dedicado en su  mayor�a al aprendizaje de \href{http://www.cs.fsu.edu/~engelen/soap.html}{GSOAP}, que es un \textit{framework} para poder programar servidores en c�digo C y C#. El estudi� de esta herramienta llev� bastante tiempo debido a que su documentaci�n, no solo estaba puramente en ingl�s, sino que adem�s no era lo suficientemente precisa como para alguien que estuviera empezando a programar servicios WEB.

En el \textit{Sprint Planning} se propuso como objetivos a tener en el siguiente \textit{sprint}:
\begin{itemize}
	\item Evaluar si usar Axiss2/c o GSOAP para el servidor
	\item Continuar con la documentaci�n
	\item Ejemplos de Android
\end{itemize}

En este \textit{sprint} podemos identificar las siguientes tareas que finalmente se definieron en la herramienta VersionOne:
\begin{itemize}
	\item Pruebas en Linux: Se hacen pruebas de la librer�a GSOAP en la m�quina virtual de Linux, en su distribuci�n Ubuntu.
	\item Pruebas en Android: Se hacen pruebas de distintos c�digos y aplicaciones en Android que posteriormente nos puedan servir
	\item Primera versi�n del Cliente: Se genera la primera versi�n del cliente Android para el proyecto. Esta s�lo es una interfaz sencilla que toma una foto con el m�vil y la guarda.
	\item Estudio de WSDL y SOAP: Se estudia estas dos especificaciones de XML, junto con un peque�o repaso de XML. Estas dos especificaciones son totalmente necesarias para el desarrollo de la aplicaci�n, aunque al final puede que no se lleguen a usar directamente, el conocimiento de las mismas es requisito indispensable para conocer c�mo trabaja el servidor internamente.
	\item Descargar materiales: Para el estudio y trabajo con la librer�a GSOAP es necesario una serie de archivos, esta tarea es en la que nos descargamos todos los necesarios.
	\item Primera versi�n del fichero WSDL: Un versi�n inicial de un fichero WSDL con el que generar archivos \textit{stubs} con la herramienta GSOAP. Este fichero contiene las especificaciones de los servicios que dar� el servidor al cliente y qu� tipo de datos admite y devuelve.
	\item Generar documentaci�n:Como en todas las iteraciones se procura generar la documentaci�n asociada.
\end{itemize}

En el \textit{Retrospective Meeting} se determina qu� tareas del \textit{sprint} han sido terminadas, cu�les no y el avance de la iteraci�n. En este caso se identific� lo siguiente:
\begin{itemize}
	\item Se decidi� usar GSOAP porque era m�s sencillo.
	\item Se estudi� GSOAP y se consigui� implementar un cliente en Linux de un ejemplo b�sico de calculadora.
	\item Se continu� con ejemplos de Android creando el proyecto del cliente en su primera versi�n, aunque este no hac�a nada m�s que mostrar una imagen por pantalla.
\end{itemize}

\figura{1}{imgs/S3_1.png}{Burn-down del sprint 5}{Sprint3}{}
Esta iteraci�n tambi�n tiene un gr�fico de tipo \textit{burn-down} asociado y se puede ver en la imagen \ref{Sprint3}

\subsubsection{Sprint 6: 28 de Marzo al 22 de Abril}
En esta iteraci�n se procedi� a crear prototipos, tanto de cliente como de servidor, del proyecto para poder empezar con el desarrollo software del mismo.

En la planificaci�n dentro del \textit{Sprint Planning} se determin�:
\begin{itemize}
	\item Tratar de implementar el ejemplo b�sico de servidor, con el que podamos hacer las primeras pruebas.
	\item Tratar de hacer el cliente en Android, que se conectara al ejemplo b�sico de servido. Con esto empezaremos a establecer c�mo se realizar� la conexi�n entre cliente y servidor.
	\item Tratar de hacer el prototipo con DeepBeliefSDK para empezar a intentar conectarlo con el servidor.
	\item Continuar con la documentaci�n.
\end{itemize}

Esos fueron los objetivos fijados para este \textit{sprint}, que es inusualmente largo debido a la serie de problemas que se encontraron en el mismo y que son explicados enn el \textit{Retrospective Meeting}.

Mientras que en VersionOne lo que tenemos definido es:
\begin{itemize}
	\item Instalar Apache y dem�s herramientas: Nos encontramos con que el servidor de GSOAP necesitaba de Apache para poder ejecutarlo en localhost y as� establecer la conexi�n con el cliente Android, as� que procedimos a la instalaci�n de las herramientas necesarias.
	\item Construcci�n del servidor: Se mont� el servidor con un ejemplo de calculadora b�sico y se comprob� con las propias herramientas de GSOAP, que este estaba bien definido.
	\item Construcci�n del cliente: Se mont� un cliente GSOAP con el que tambi�n se probo el ejemplo b�sico de servidor, este y el servidor funcionaban de manera adecuada.
	
	\item Pruebas con distintos servicios en Linux: Se remontaron varias veces el cliente y el servidor, pero con distintas especificaciones SOAP, para probar varios tipos de datos en el env�o.
	\item Primera versi�n del Servidor: Despu�s de construir el ejemplo b�sico se paso a conectar el DeepBeliefSDK con el servidor y este con Apache. Se empez� por probar la conexi�n con Apache, la cu�l result� un rotundo fracaso.
	\item Escritura de la documentaci�n: Se genera, como siempre, la documentaci�n asociada a la iteraci�n.
\end{itemize}
Finalmente tenemos nuestro \textit{Retrospective Meeting}, en el que determinamos lo siguiente:
\begin{itemize}
	\item No se consigui� conectar GSOAP con Apache
	\item La documentaci�n para el proceso de conectar Apache y GSOAP es escasa y la poca que hay no nos ense�a un procedimiento correcto para hacer funcionar esto.
	\item Se decide abandonar la programaci�n en C y la librer�a DeepBeliefSDK
	\item Conclusi�n: No se consiguieron los objetivos y se decide tomar un nuevo rumbo con el proyecto.
\end{itemize}
\figura{1}{imgs/S4_1.png}{Burn-down del sprint 6}{Sprint4}{}
\paragraph{}Podemos observar el gr�fico de la imagen \ref{Sprint2}, en el que observamos el gr�fico \textit{burn-down} del \textit{sprint}.
%
%\portadasAuxiliares{Anexo II - Especificaci�n de requisitos}
%\chapter{Especificaci�n de requisitos}
%\section{Introducci�n}
En este anexo se presentar�n los distintos requisitos funcionales de la aplicaci�n de ejemplo que se ha desarrollado. Se expondr�n tanto los requisitos como los diagramas de caso de uso, plantillas de caso de uso.
\section{Requisitos Funcionales}
Los requisitos que se establecen son las funcionalidades m�nimas que se han propuesto como objetivos en la aplicaci�n.
\subsection{Requisitos Funcionales en el Servidor}
Estos son los requisitos que se han establecido para que el servidor se considere que cumple con la funcionalidad exigida.
\begin{itemize}
	\item \textbf{RF1 Recibir Imagen:} El servidor deber� ser capaz de, a trav�s de una petici�n de tipo POST, recibir una imagen correctamente y guardarla en el sistema. Cuando este reciba una petici�n de tipo POST proceder� a recibir los datos, seguidamente se guardar�n los datos en el sistema de almacenamiento de la m�quina en la que este alojado el servidor. Adem�s deber� comprobar que lo que est� recibiendo es una imagen para evitar que se nos env�e archivos maliciosos con extensiones no permitidas.
	\item \textbf{RF2 Procesar Imagen:} El servidor ejecutar� la herramienta de predicci�n sobre los datos recibidos previamente. Deber� tener la configuraci�n adecuada para que la herramienta funcione de manera adecuada y que ejecute una predicci�n sobre la imagen. La dificultad se centra en la configuraci�n de una herramienta tan compleja como puede ser las de tratamiento de im�genes en \textit{machine learning}.
	\item \textbf{RF3 Devolver Predicci�n:} El servidor deber� ser capaz de devolver una frase en espa�ol que sea el resultante de la predicci�n sobre la imagen. Para ello tiene otro requisito funcional subyacente a �l.
	\begin{itemize}
		\item \textbf{RF3.1 Traducir Predicci�n:} El servidor deber� ser capaz de traducir la predicci�n del ingl�s al espa�ol, ya que las herramientas que se pueden utilizar para esta aplicaci�n trabajan todas en ingl�s. 
	\end{itemize}
\end{itemize}
\subsection{Requisitos Funcionales en el Cliente}
Se presentan ahora los requisitos que se han establecido en el cliente.
\begin{itemize}
	\item \textbf{RF4 Solicitar Preedicci�n:} El cliente Android deber� permitir al usuario tomar una foto. Teniendo en cuenta que la persona a la que va dirigida la aplicaci�n es una persona con dificultades de visi�n o invidente, esta deber� permitir al usuario tomar la foto sin que este deba tener conocimiento alguno de la interfaz o de si hay un bot�n que este deba pulsar. Por tanto, la aplicaci�n mostrar� una pantalla sin interfaz que est� a la espera de que el usuario simplemente toque la pantalla, entonces la aplicaci�n tomar� autom�ticamente la foto.
	\begin{itemize}
		\item \textbf{RF4.1 Tomar Foto:} La aplicaci�n tomar� un foto de manera autom�tica sin necesidad de que el usuario tenga que interactuar con la interfaz de manera espec�fica, sino que podr� hacerlo sin saber qu� se encuentra en la pantalla, tan s�lo deber� tocar cualquier punto de la pantalla.
		\item \textbf{RF4.2 Mandar Foto:} La aplicaci�n cliente deber� ejecutar una petici�n de tipo POST hacia un servidor y mandar correctamente la imagen para que esta sea procesada por el servidor. Mientras el servidor procesa y recibe la imagen, la aplicaci�n mostrar� una ventana que informe de de la imagen se esta procesando, esta ventana no es con el objetivo de que el usuario la vea, sino de que la aplicaci�n este a la espera de que la imagen sea procesada y que el usuario no tenga la capacidad de realizar nada en ese per�odo. Esto asegurar� que la aplicaci�n no falle mientras se procesa la imagen.
		\item \textbf{RF4.3 Leer Predicci�n:} Una vez se ha procesado la imagen la aplicaci�n deber� ser capaz de recibir la predicci�n del servidor y, debido a que la persona que la use no tendr� la capacidad de leerla, la aplicaci�n leer� la aplicaci�n en alto a trav�s de una librer�a \textit{text to speech}.
	\end{itemize}
\end{itemize}
\section{Diagrama de casos de uso}
En este apartado se procede a presentar el diagrama o diagramas de caso de uso de la aplicaci�n, adem�s de las correspondientes tablas asociadas a cada uno de los requisitos funcionales del sistema.

El diagrama de casos de uso se puede ver en la imagen \ref{CasoUso1}.

Mientras que las tablas de cada uno de los requisitos funcionales se pueden ver en:
\begin{itemize}
	\item \textbf{RF1 Recibir Imagen}: Se puede ver en la tabla \ref{tabla:RF1}.
	\item \textbf{RF2 Procesar Imagen:} Se puede ver en la tabla \ref{tabla:RF2}.
	\item \textbf{RF3 Devolver Predicci�n:} Se puede ver en la tabla \ref{tabla:RF3}.
	\begin{itemize}
		\item \textbf{RF3.1 Traducir Predicci�n:} Se puede ver en la tabla \ref{tabla:RF3.1}.
	\end{itemize}
\end{itemize}
\figura{1}{imgs/CasoUso1.png}{Diagrama de caso de uso del RF4}{CasoUso1}{}
\tablaSmallSinColores{Caso de uso: RF1 Recibir imagen}{p{3cm} p{.75cm} p{9.5cm}}{RF1}{
  \multicolumn{3}{l}{RF1  Recibir imagen} \\
 }
 {
  Descripci�n                            & \multicolumn{2}{l}{Se recibir� una imagen enviada desde un cliente y se guardar�.} \\\hline
  Precondiciones                         & \multicolumn{2}{l}{Deber� haberse recibido una petici�n de tipo POST} \\\hline
  \multirow{4}{3.5cm}{Secuencia normal}  & Paso & Acci�n \\\cline{2-3}
                                         & 1    & Se detecta una petici�n POST \\\cline{2-3}
                                         & 2    & Se comprueba que el fichero recibido es una imagen \\\cline{2-3}
                                         & 3    & Si es imagen, se guarda, sino, se descarta. \\\hline
  Postcondiciones                        & \multicolumn{2}{l}{Si se ha recibido una imagen, esta deber� estar guardada} \\\hline
  \multirow{2}{3.5cm}{Excepciones}       & Paso & Acci�n \\\cline{2-3}
                                         &      &  \\\hline
  Rendimiento                            &      & \\\hline
  Frecuencia                             & \alta{2} \\\hline
  Importancia                            & \alta{2} \\\hline
  Urgencia                               & \alta{2} \\\hline
  Comentarios                            & \multicolumn{2}{l}{S�lo se reciben im�gennes tipo: jpg, jpeg y png} \\
}
\tablaSmallSinColores{Caso de uso: RF2 Procesar imagen}{p{3cm} p{.75cm} p{9.5cm}}{RF2}{
  \multicolumn{3}{l}{RF2  Procesar imagen} \\
 }
 {
  Descripci�n                            & \multicolumn{2}{l}{La imagen ser� procesada a trav�s de la herramienta elegida.} \\\hline
  Precondiciones                         & \multicolumn{2}{l}{Se deber� haber guardado una imagen recibida por el usuario} \\\hline
  \multirow{4}{3.5cm}{Secuencia normal}  & Paso & Acci�n \\\cline{2-3}
                                         & 1    & Se escribe nombre de la imagen en fichero de procesado ``tasks.txt'' \\\cline{2-3}
                                         & 2    & Se ejecuta \textit{script} de  extracci�n de caracter�sticas de la imagen \\\cline{2-3}
                                         & 3    & Se lanza la extracci�n de frase. \\\hline
  Postcondiciones                        & \multicolumn{2}{l}{Se tiene un documento ``result.html'' con la predicci�n} \\\hline
  \multirow{2}{3.5cm}{Excepciones}       & Paso & Acci�n \\\cline{2-3}
                                         &      &  \\\hline
  Rendimiento                            &      & \\\hline
  Frecuencia                             & \alta{2} \\\hline
  Importancia                            & \alta{2} \\\hline
  Urgencia                               & \media{2} \\\hline
  Comentarios                            & \multicolumn{2}{l}{El fichero tiene dentro la frase que se quiere obtener} \\
}
\tablaSmallSinColores{Caso de uso: RF3 Devolver predicci�n}{p{3cm} p{.75cm} p{9.5cm}}{RF3}{
  \multicolumn{3}{l}{RF3 Devolver predicci�n} \\
 }
 {
  Descripci�n                            & \multicolumn{2}{l}{Se recoger� la predicci�n sobre la imagen y como respuesta a la petici�n esta ser� devuelta.} \\\hline
  \multirow{3}{*}{Precondiciones}		& 1	&Se deber� haber guardado una imagen recibida por el usuario \\\cline{2-3}
  										& 2 & Deber� existir el documento ``result.html'' \\\cline{2-3}
  										& 3 & Deber� traducirse la cadena al espa�ol \\\hline
  \multirow{5}{3.5cm}{Secuencia normal}  & Paso & Acci�n \\\cline{2-3}
                                         & 1    & Se abre el fichero ``result.html'' si existe. \\\cline{2-3}
                                         & 2    & Se extrae la cadena en ingl�s. \\\cline{2-3}
                                         & 3    & Se Traduce la cadena al espa�ol \\\cline{2-3}
                                         & 4	& Se devuelve la cadena como resultado de la petici�n POST \\\hline
  Postcondiciones                        & \multicolumn{2}{l}{Se ha devuelto una cadena de texto, correspondiente a la predicci�n.} \\\hline
  \multirow{2}{3.5cm}{Excepciones}       & Paso & Acci�n \\\cline{2-3}
                                         &  1   & Si el fichero ``result.html'' no existe se genera un error. \\\hline
  Rendimiento                            &      & \\\hline
  Frecuencia                             & \alta{2} \\\hline
  Importancia                            & \alta{2} \\\hline
  Urgencia                               & \media{2} \\\hline
  Comentarios                            &  \\
}
\tablaSmallSinColores{Caso de uso: RF3.1 Traducir predicci�n}{p{3cm} p{.75cm} p{9.5cm}}{RF3.1}{
  \multicolumn{3}{l}{RF3.1 Traducir predicci�n} \\
 }
 {
  Descripci�n                            & \multicolumn{2}{l}{A partir de una cadena de caracteres en ingl�s, se traducir� al espa�ol.} \\\hline
  \multirow{2}{*}{Precondiciones}		& 1	& La cadena no deber� ser nula\\\cline{2-3}
  										& 2 & La cadena deber� estar en ingl�s \\\hline
  \multirow{6}{3.5cm}{Secuencia normal}  & Paso & Acci�n \\\cline{2-3}
                                         & 1    & Se crea url con cabecera para la traducci�n. \\\cline{2-3}
                                         & 2    & Se manda petici�n a la p�gina de Google traductor. \\\cline{2-3}
                                         & 3    & Se reciben los datos en un json \\\cline{2-3}
                                         & 4	& Se extrae la cadena del json\\\cline{2-3}
                                         & 5	& Se devuelve la cadena traducida\\\hline
  Postcondiciones                        & \multicolumn{2}{l}{Se ha devuelto una cadena de texto, correspondiente a la predicci�n.} \\\hline
  \multirow{2}{3.5cm}{Excepciones}       & Paso & Acci�n \\\cline{2-3}
                                         &      & \\\hline
  Rendimiento                            &      & \\\hline
  Frecuencia                             & \alta{2} \\\hline
  Importancia                            & \media{2} \\\hline
  Urgencia                               & \baja{2} \\\hline
  Comentarios                            &  \multicolumn{2}{l}{Se simula un conexi�n v�a navegador para la traducci�n}\\
}
%
%\portadasAuxiliares{Anexo III - Especificaci�n de dise�o}
%\chapter{Especificaci�n de dise�o}
%\section{Introducci�n}
En este apartado se proceder� a explicar las especificaciones de dise�o que se han ido utilizando para el desarrollo de la aplicaci�n.


\section{Dise�o en el Servidor}
En primer lugar se introducir� las especificaciones que se han seguido para desarrollar el servidor.
\subsection{API RESTful}
Se ha construido el servidor de manera que siga, en la medida de lo posible, las especificaciones de la API RESTful. Esto se puede observar en el hecho de que se ha usado Flask para programar el servidor, que es un \textit{framework} que permite la programaci�n de peticiones de la API RESTful de manera muy sencilla. Adem�s, tiene soporte para el uso de urls a la hora de trabajar con peticiones y funciones.

En el servidor se ha trabajado s�lo con una url, la url principal del servidor. En la imagen (\ref{lst:URL}) se puede observar c�mo se ha definido la url base para el servidor. Adem�s podemos ver que esta soporta dos tipos de m�todos, o bien la petici�n de tipo GET, o bien la petici�n de tipo POST.
\begin{lstlisting}[frame=none,language=Python,caption={Url �nica y principal del servidor},basicstyle=\large,label={lst:URL}]
@app.route("/", methods=['GET', 'POST'])
def index():
\end{lstlisting}
Si nos fijamos, tenemos una funci�n justo debajo de la notaci�n ``app.route'', esto implica que al acceder a esa url, se ejecuta esta funci�n y es ah� donde se determina qu� hace el servidor en cada caso.

En el servidor, la funci�n trata de manera predeterminada la petici�n como una petici�n de tipo GET, as� que habr� que tratar el m�todo POST de alguna manera (\ref{lst:POST}).
\begin{lstlisting}[frame=none,language=Python,caption={Tratando el m�todo POST},basicstyle=\large,label={lst:POST}]
if request.method == 'POST':
	#Aqu� lo que se har�a si es m�todo POST
	return "la respuesta de la operaci�n"
#Aqu� lo que se har�a si es m�todo GET
return "respuesta de un m�todo GET"
\end{lstlisting}
EL objeto ``request'' es un objeto que usa Flask internamente y representa la petici�n que se est� recibiendo. Para acceder a qu� tipo de petici�n es, accedemos a su variable ``method'' en la que obtendremos la respuesta y con este valor se trabaja la petici�n POST.

Para asegurar la seguridad del sistema se ha usado una lista (\ref{lst:TIPOS}) de tipo de archivos admisibles por el servidor, lo que impide que se nos mande archivos maliciosos con extensiones extra�as, pues si el archivo tiene una extensi�n incorrecta este descarta la petici�n.
\begin{lstlisting}[frame=none,language=Python,caption={Tipos admitidos por el servidor},basicstyle=\large,label={lst:TIPOS}]
ALLOWED_EXTENSIONS = set(['jpg','JPG','jpeg','JPEG'])
\end{lstlisting}
Posteriormente se define una funci�n que se encarga de comprobar si el tipo es uno de los permitidos y si se cumple la condici�n, entonces se contin�a el proceso, de lo contrario se descarta el archivo recibido.
\begin{lstlisting}[frame=none,language=Python,caption={Comprobando tipos},basicstyle=\large,label={lst:FLAG}]
def allowed_file(filename):
    return '.' in filename and \
           filename.rsplit('.', 1)[1] in ALLOWED_EXTENSIONS
\end{lstlisting}

%
%\portadasAuxiliares{Anexo IV - Manual del programador}
%\chapter{Manual del programador}
%\label{anx:ManualProgramador}
En esta secci�n se proceder� a la explicaci�n detallada de c�mo instalar las herramientas necesarias y qu� herramientas son necesarias para trabajar sobre este proyecto.
\section{Instalaci�n del JDK}
La primera, y m�s esencial de las herramientas, es el JDK de java, que es el set o conjunto de herramientas y librer�as para los desarolladores de java.

\paragraph{} Primero deberemos ir a la p�gina de Oracle\footnote{\url{http://www.oracle.com/technetwork/java/javase/downloads/jdk8-downloads-2133151.html?ssSourceSiteId=otnes}} en la que descargaremos el jdk, la p�gina deber�a tener un aspecto m�s o menos como el de la imagen \ref{figImagen1}
En dicha p�gina tendremos que aceptar la licencia y posteriormente descargar el JDK que sirva para nuestra m�quina. Una vez hemos descargado dicho archivo, lo ejecutamos. Una vez ejecutado seguimos los siguientes pasos para su instalaci�n.
\figura{1}{imgs/jdk_1.png}{P�gina de descarga del JDK de Java}{figImagen1}{}\\
  \begin{figure}[]
    \begin{center}%
      \Ovalbox{%
        \begin{minipage}{\anchoFloat}%
          \begin{center}%
            \begin{tabular}{ l r}
 				\includegraphics[width=0.45\textwidth]{imgs/jdk_2.png} & 
 				\includegraphics[width=0.45\textwidth]{imgs/jdk_3.png} \\
			\end{tabular}
            \caption{Instalaci�n del JDK, paso 1.}%
            \label{figImagen2}%
          \end{center}%
        \end{minipage}
      }%
    \end{center}%
  \end{figure}%
    \begin{figure}[]
    \begin{center}%
      \Ovalbox{%
        \begin{minipage}{\anchoFloat}%
          \begin{center}%
            \begin{tabular}{ l r}
 				\includegraphics[width=0.45\textwidth]{imgs/jdk_4.png} & 
 				\includegraphics[width=0.45\textwidth]{imgs/jdk_5.png} \\
			\end{tabular}
            \caption{Instalaci�n del JDK, paso 2.}%
            \label{figImagen3}%
          \end{center}%
        \end{minipage}
      }%
    \end{center}%
  \end{figure}%
En la imagen \ref{figImagen2} vemos el primer paso para la instalaci�n del \textit{JDK}, el cual consta de dos pantallas. La primera ser� solamente una pantalla de bienvenida, por lo que debemos pulsar a \textit{next} o siguiente. La segunda pantalla nos muestra los elementos que van a ser instalados, esto no lo tocamos; adem�s nos muestra tambi�n el directorio en el que queremos instalar el \textit{JDK}, esto lo podemos dejar como est� o podemos, si queremos, configurarlo en un directorio personal que queramos. Lo �nico es que habr� que tener cuidado con el \textit{PATH} del equipo para que luego \textit{Android Studio} pueda encontrar la distribuci�n de \textit{JDK} que tengamos en nuestra m�quina.
\paragraph{}En la imagen \ref{figImagen3} nos encontramos el paso 2 para la instalaci�n de nuestro \textit{JDK}, este tambi�n consta de dos pantallas. La primera es el estado de la instalaci�n, veremos una barra que se ir� llenando en funci�n de que la instalaci�n vaya avanzando. Finalmente se nos mostrar� la pantalla que nos indicar� que la instalaci�n se ha realizado correctamente y nos da la opci�n de pulsar el bot�n \textit{Next Steps} uqe nos llevar� a la \textit{API de Java} o simplemente finalizar la instalaci�n.
\figura{1}{imgs/sdk_1.png}{P�gina de descarga de Android Studio}{figImagen4}{}\\
\section{Instalaci�n de Android Studio}
\paragraph{}Para la programaci�n del cliente se usar� \textit{Android Studio}, una herramienta ideada para programar en \textit{Android} exclusivamente. Lo primero que debemos hacer es dirigirnos a la p�gina web donde descargaremos Android Studio\footnote{\url{http://developer.android.com/sdk/index.html}}, la cual tendr� un aspecto similar al de la \ref{figImagen4}. En ella haremos \textit{click} sobre el bot�n de descarga (\textit{Download Android Studio}) y procederemos a la instalaci�n ejecutando el fichero que hemos descargado, todo este proceso es para un sistema operativo \textit{Windows}.
    \begin{figure}[]
    \begin{center}%
      \Ovalbox{%
        \begin{minipage}{\anchoFloat}%
          \begin{center}%
            \begin{tabular}{ l r}
 				\includegraphics[width=0.45\textwidth]{imgs/sdk_2.png} & 
 				\includegraphics[width=0.45\textwidth]{imgs/sdk_3.png} \\
			\end{tabular}
            \caption{Instalaci�n de Android Studio, paso 1.}%
            \label{figImagen5}%
          \end{center}%
        \end{minipage}
      }%
    \end{center}%
  \end{figure}%
      \begin{figure}[]
    \begin{center}%
      \Ovalbox{%
        \begin{minipage}{\anchoFloat}%
          \begin{center}%
            \begin{tabular}{ l r}
 				\includegraphics[width=0.45\textwidth]{imgs/sdk_4.png} & 
 				\includegraphics[width=0.45\textwidth]{imgs/sdk_5.png} \\
			\end{tabular}
            \caption{Instalaci�n de Android Studio, paso 2.}%
            \label{figImagen6}%
          \end{center}%
        \end{minipage}
      }%
    \end{center}%
  \end{figure}%
      \begin{figure}[]
    \begin{center}%
      \Ovalbox{%
        \begin{minipage}{\anchoFloat}%
          \begin{center}%
            \begin{tabular}{ l r}
 				\includegraphics[width=0.45\textwidth]{imgs/sdk_6.png} & 
 				\includegraphics[width=0.45\textwidth]{imgs/sdk_7.png} \\
			\end{tabular}
            \caption{Instalaci�n de Android Studio, paso 3.}%
            \label{figImagen7}%
          \end{center}%
        \end{minipage}
      }%
    \end{center}%
  \end{figure}%
      \begin{figure}[]
    \begin{center}%
      \Ovalbox{%
        \begin{minipage}{\anchoFloat}%
          \begin{center}%
            \begin{tabular}{ l r}
 				\includegraphics[width=0.45\textwidth]{imgs/sdk_8.png} & 
 				\includegraphics[width=0.45\textwidth]{imgs/sdk_9.png} \\
			\end{tabular}
            \caption{Instalaci�n de Android Studio, paso 4.}%
            \label{figImagen8}%
          \end{center}%
        \end{minipage}
      }%
    \end{center}%
  \end{figure}%
      \begin{figure}[]
    \begin{center}%
      \Ovalbox{%
        \begin{minipage}{\anchoFloat}%
          \begin{center}%
            \begin{tabular}{ l r}
 				\includegraphics[width=0.45\textwidth]{imgs/sdk_10.png} & 
 				\includegraphics[width=0.45\textwidth]{imgs/sdk_11.png} \\
			\end{tabular}
            \caption{Instalaci�n de Android Studio, paso 5.}%
            \label{figImagen9}%
          \end{center}%
        \end{minipage}
      }%
    \end{center}%
  \end{figure}%
  
\paragraph{} En el primer paso de la instalaci�n, que se puede observar en la imagen \ref{figImagen5}, nos encontraremos con dos pantallas. La primera es simplemente una pantalla de bienvenida a la instalaci�n de nuestro \textit{Android Studio}, debemos pulsar al bot�n \textit{next} para empezar nuestra instalaci�n. Entonces pasamos a la siguiente pantalla y nos da la opci�n de elegir qu� elementos instalar, nosotros hemos dejado todos marcados pero puede que algunos no sean obligatoriamente necesarios. Una vez seleccionados los componentes de nuestra instalaci�n pasamos a dar el bot�n \textit{next} de la pantalla, entonces saltamos al paso 2.
\paragraph{}El segundo paso de nuestra instalaci�n lo podemos observar en la imagen \ref{figImagen6}. Una vez hemos pasado la selecci�n de componentes en nuestro \textit{Android Studio}, tenemos que aceptar la licencia del software, pasaremos a leer la licencia, si es de nuestro inter�s, y cuando estemos de acuerdo con ella, pulsaremos el bot�n \textit{I Agree}. Esto no habr� llevado a la siguiente pantalla de la instalaci�n, en la que podremos elegir los directorios en los que instalaremos nuestros componentes, nosotros los hemos dejado por defecto aunque se pueden personalizar en funci�n de las necesidades del usuario. Una vez establecida la configuraci�n de directorios en la instalaci�n, pulsaremos el bot�n \textit{next}.
\paragraph{}Pasaremos entonces al tercer paso de la instalaci�n, la cual puede ser vista en la imagen \ref{figImagen7}. En la primera pantalla de este paso tendremos que elegir la cantidad de memoria \textit{RAM} que asignaremos a nuestro \textit{Android Studio}, podemos dejar la cantidad recomendada o podemos asignarle m�s cantidad, si disponemos de ella, para que tenga un funcionamiento m�s fluido. Una vez establecido esto, se pulsara el bot�n \textit{next}. En nuestra segunda pantalla nos pregunta el directorio en el que se iniciar� la aplicaci�n y puedes poner la opci�n de no crear un acceso directo, nosotros hemos dejado la configuraci�n por defecto. Finalmente pulsamos el bot�n \textit{Install}.
\paragraph{}Ahora en el paso 4, el cual est� representado en la imagen \ref{figImagen8}, solamente tendremos que observar c�mo avanza la instalaci�n, esto se nos ir� mostrando a trav�s de una barra de progreso. Una vez se la barra se ha llenado podremos hacer \textit{click} en el bot�n \textit{Next} para pasar al �ltimo paso de nuestra instalaci�n.
\paragraph{}En nuestro quinto y �ltimo paso, el cual podemos ver en la imagen \ref{figImagen9}, se nos informar� de que la instalaci�n ha terminado y tendremos la opci�n de iniciar nuestro \textit{Android Studio} nada m�s acabar la instalaci�n. Entonces pulsamos al bot�n \textit{Finish}, se nos abrir� nuestro \textit{Android Studio} y con esto la instalaci�n se considera terminada.
\section{SDK Manager, instalando herramientas}
      \begin{figure}[]
    \begin{center}%
      \Ovalbox{%
        \begin{minipage}{\anchoFloat}%
          \begin{center}%
            \begin{tabular}{ l r}
 				\includegraphics[width=0.45\textwidth]{imgs/mng1.png} & 
 				\includegraphics[width=0.45\textwidth]{imgs/mng2.png} \\
			\end{tabular}
            \caption{SDK Manager paso 1.}%
            \label{mng1}%
          \end{center}%
        \end{minipage}
      }%
    \end{center}%
  \end{figure}%
  \figura{1}{imgs/mng3.png}{SDK Manager paso 2.}{mng2}{}
Para que la futura importaci�n del proyecto se haga de manera correcta hay que instalar una serie de herramientas, estas pueden ser instaladas a trav�s del SDK \textit{Manager} del Android Studio. Para abrir el SDK \textit{Manager}, debemos hacer click sobre el bot�n resaltado en la primera imagen (\ref{mng1}). Seguidamente se nos abrir� una ventana gestora de paquetes de instalaci�n, que  se puede ver en la segunda imagen (\ref{mng1}). Entonces se seleccionan los paquetes a instalar, se abrir� una nueva ventana (\ref{mng2}) para aceptar las licencias, las aceptas y le das a instalar.

Los paquetes que necesitaremos ser�m:
\begin{itemize}
	\item Android SDK Tools
	\item Android SDK Platform-tools
	\item Android SDK Build-tools
	\item Todo el paquete Android 5.1.1(API 22)
	\item Android Support Repository
	\item Android Support Library
	\item Google Repository
	\item Google USB Driver
	\item Google Web Driver
\end{itemize}
\section{Importando Cliente en Android Studio}
\figura{1}{imgs/Import1.png}{Importando proyecto en Android Studio}{import1}{}
\figura{1}{imgs/Import2.png}{Importando proyecto en Android Studio}{import2}{}
\figura{0.60}{imgs/Import3.png}{Importando proyecto en Android Studio}{import3}{}
\figura{1}{imgs/Import4.png}{Importando proyecto en Android Studio}{import4}{}
Se procede a explicar la manera de importar el proyecto en Android Studio, que es la aplicaci�n con la que se ha trabajado y con la que mejor se puede trabajar sobre el proyecto. Primero deberemos iniciar el Android Studio (\ref{import1}), seguidamente le damos a \textit{File} o Archivo -> \textit{Open\dots } o Abrir\dots (\ref{import2}). Entonces se nos abrir� una ventana (\ref{import3}) en la que deberemos navegas hasta la carpeta en la que tenemos almacenado el proyecto, seleccionamos el archivo con nombre Imagen y que tiene el icono de Android Studio. Entonces le damos a aceptar y el programa proceder� a importar el s�lo todo el proyecto (\ref{import4}), al finalizar podremos ponernos enseguida a trabajar.

\section{Instalaci�n de Matlab}
\label{MATLAB}
Para poder ejecutar el \textit{script} de instalaci�n del servidor, el cu�l se instala en Ubuntu, debemos tener como pre-instalado Matlab. Para ello necesitaremos una licencia, en este proyecto se ha trabajado con la licencia de estudiante de la UBU.

Al trabajar en Ubuntu de 64 bit, hemos descargado el instalador para Linux de Matlab. Hay que recordar que Caffe s�lo trabaja con ciertas versiones de Matlab a la hora de descargar el instalador (2014a/b, 2013a/b, and 2012b). Una vez se ha descargado el instalador se tiene que ejecutar el archivo\textit{install} del conjunto de archivos que hemos descargado (\ref{lst:install}).
\begin{lstlisting}[frame=none,language=bash,caption={Ejecutando \textit{install} de Matlab},basicstyle=\large,label={lst:install}]
sudo ./install
\end{lstlisting}
Entonces nos aparecer� una ventana a trav�s de la cual guiaremos nuestra instalaci�n.

Con nuestra licencia, la instalaci�n ser�a. Primero debemos elegir la primera opci�n de la primera imagen y despu�s aceptar la licencia de la segunda imagen en el paso 1 (\ref{mat1}). 

Seguidamente, en el segundo paso de instalaci�n. deberemos aportar nuestras credenciales, nuestra cuenta de MathWorks, como se muestra en la primera imagen y despu�s seleccionar el lugar de instalaci�n, tal y como se muestra en la segunda imagen del paso 2 (\ref{mat2}). Se recomienda dejar la carpeta de destino de la instalaci�n a la que viene por defecto, porque evitar� futuros problemas de configuraci�n cuando instalemos el servidor y sus dependencias.

En el tercer paso de la instalaci�n decidiremos si creamos un link para invocar a Matlab desde el terminal, esto se recomienda hacerlo y dejar el link en la ruta por defecto del instalador, como podemos ver en la primera imagen. Despu�s se nos pedir� seleccionar los paquetes que queremos instalar a Matlab, seleccionaremos los b�sicos, como se ve en la imagen dos del tercer paso (\ref{mat3}).

En el cuarto paso se nos pide la confirmaci�n de los paquetes que hemos solicitado instalar en el paso previo, esto puede verse en la primera imagen; comprobaremos que tenemos los paquetes que deseamos y le damos a confirmar. En la segunda imagen vemos la barra de progreso de la instalaci�n que se nos muestra mientras Matlab es instalado (\ref{mat4}).

En el quinto y �ltimo paso se te muestra la ventana de instalaci�n satisfactoria y te comunica que debes activar Matlab para poder usarlo. Dejamos la opci�n de ``Activar MATLAB'' marcada y le damos a siguiente. As� concluimos la instalaci�n de Matlab, pero se tiene que activar antes (\ref{mat5}).

  \begin{figure}[]
    \begin{center}%
      \Ovalbox{%
        \begin{minipage}{\anchoFloat}%
          \begin{center}%
            \begin{tabular}{ l r}
 				\includegraphics[width=0.45\textwidth]{imgs/matlab1.png} & 
 				\includegraphics[width=0.45\textwidth]{imgs/matlab2.png} \\
			\end{tabular}
            \caption{Instalaci�n de Matlab paso 1.}%
            \label{mat1}%
          \end{center}%
        \end{minipage}
      }%
    \end{center}%
  \end{figure}%
  \begin{figure}[]
    \begin{center}%
      \Ovalbox{%
        \begin{minipage}{\anchoFloat}%
          \begin{center}%
            \begin{tabular}{ l r}
 				\includegraphics[width=0.45\textwidth]{imgs/matlab3.png} & 
 				\includegraphics[width=0.45\textwidth]{imgs/matlab4.png} \\
			\end{tabular}
            \caption{Instalaci�n de Matlab paso 2.}%
            \label{mat2}%
          \end{center}%
        \end{minipage}
      }%
    \end{center}%
  \end{figure}%
  \begin{figure}[]
    \begin{center}%
      \Ovalbox{%
        \begin{minipage}{\anchoFloat}%
          \begin{center}%
            \begin{tabular}{ l r}
 				\includegraphics[width=0.45\textwidth]{imgs/matlab5.png} & 
 				\includegraphics[width=0.45\textwidth]{imgs/matlab6.png} \\
			\end{tabular}
            \caption{Instalaci�n de Matlab paso 3.}%
            \label{mat3}%
          \end{center}%
        \end{minipage}
      }%
    \end{center}%
  \end{figure}%
  \begin{figure}[]
    \begin{center}%
      \Ovalbox{%
        \begin{minipage}{\anchoFloat}%
          \begin{center}%
            \begin{tabular}{ l r}
 				\includegraphics[width=0.45\textwidth]{imgs/matlab7.png} & 
 				\includegraphics[width=0.45\textwidth]{imgs/matlab8.png} \\
			\end{tabular}
            \caption{Instalaci�n de Matlab paso 4.}%
            \label{mat4}%
          \end{center}%
        \end{minipage}
      }%
    \end{center}%
  \end{figure}%
 \figura{1}{imgs/matlab9.png}{Instalaci�n de Matlab paso 5}{mat5}{}
 \subsection{Activar MATLAB}
 \label{ActMATLAB}
  \begin{figure}[]
    \begin{center}%
      \Ovalbox{%
        \begin{minipage}{\anchoFloat}%
          \begin{center}%
            \begin{tabular}{ l r}
 				\includegraphics[width=0.45\textwidth]{imgs/Activation1.png} & 
 				\includegraphics[width=0.45\textwidth]{imgs/Activation2.png} \\
			\end{tabular}
            \caption{Activaci�n de Matlab paso 1.}%
            \label{act1}%
          \end{center}%
        \end{minipage}
      }%
    \end{center}%
  \end{figure}%
  \begin{figure}[]
    \begin{center}%
      \Ovalbox{%
        \begin{minipage}{\anchoFloat}%
          \begin{center}%
            \begin{tabular}{ l r}
 				\includegraphics[width=0.45\textwidth]{imgs/Activation3.png} & 
 				\includegraphics[width=0.45\textwidth]{imgs/ActivationF.png} \\
			\end{tabular}
            \caption{Activaci�n de Matlab paso 2.}%
            \label{act2}%
          \end{center}%
        \end{minipage}
      }%
    \end{center}%
  \end{figure}%
Despu�s de la instalaci�n de Matlab se debe haber recibido la solicitud de activaci�n del mismo, por lo que se nos abre una ventana de activaci�n como se puede ver en la primera imagen del paso 1(\ref{act1}). Esta ventana es s�lo informativa, por lo que daremos al bot�n de siguiente. En la segunda imagen del paso 1 podemos ver que nos pide el nombre de usuario con el que se va a querer activar Matlab, se aconseja poner el nombre de usuario normal, no root. Una vez decidido el nombre con el que se va a activar Matlab se pulsa al bot�n de siguiente.

En el transcurso del primer paso hasta el segundo paso, el activador habr� hecho algunas comprobaciones con nuestra licencia y nos muestra una ventana de confirmaci�n, comprobamos que los datos son correctos y pulsamos el bot�n confirmar, esto se puede ver en la primera imagen del paso 2 (\ref{act2}). Final mente nos muestra una ventana informativa explicando que la activaci�n ha sido un �xito, como vemos en la segunda imagen del paso 2.
\section{Instalaci�n de CUDA}
\label{CUDA}
Si no nos encontramos en una m�quina virtual y queremos instalar los \textit{drivers} de NVIDIA en nuestra m�quina junto con CUDA, entonces tenemos que seguir unos pasos especiales para hacerlos, los cu�les ser�n explicados en este apartado para el caso de querer instalar controladores de NVIDIA, de lo contrario, s�lo se tendr� que descomentar las siguientes l�neas del \textit{script} de instalaci�n del servidor y sus dependencias:
\begin{lstlisting}[frame=none,language=bash,caption={Instalaci�n de CUDA en \textit{script}},basicstyle=\large,label={lst:install}]
sudo apt-get install curl
cd ~/Downloads/
curl -O "http://developer.download.nvidia.com/compute/cuda/6_5/rel/installers/cuda_6.5.14_linux_64.run"
chmod +x cuda_6.5.14_linux_64.run
./cuda_6.5.14_linux_64.run --kernel-source-path=/usr/src/linux-headers-`uname -r`/
\end{lstlisting}

En caso de querer instalar los controladores de NVIDIA, entonces tendremos que proceder a realizar una operaci�n m�s compleja para realizar su instalaci�n, debido a que para instalar los controladores no debemos estar en modo interfaz gr�fica y tener desactivadas ciertos servicios.

Primero tendremos que desinstalar cualquier rastro de alg�n controlador de NVIDIA que tengamos previamente en nuestra m�quina, lo hacemos con el siguiente comando:
\begin{lstlisting}[frame=none,language=sh,caption={Instalaci�n \textit{driver} de NVIDIA paso 1},basicstyle=\large]
sudo apt-get remove --purge nvidia* 
\end{lstlisting}
Despu�s de esto se deber� desactivar el controlador libre ``noveau'', el cual lo desactivamos de la siguiente manera:
\begin{itemize}
	 \item Primero, se tiene que ejecutar el siguiente comando en tu shell:
	 \begin{lstlisting}[frame=none,language=sh,caption={Instalaci�n \textit{driver} de NVIDIA paso 2},basicstyle=\large]
	 sudo nano /etc/modprobe.d/blacklist.conf
	 \end{lstlisting}
	 \item Segundo, debes a�adir lo siguiente al final del fichero, que se ha abierto tras ejecutar el anterior comando, y guardar los cambios:
	 \begin{lstlisting}[frame=none,language=sh,caption={Instalaci�n \textit{driver} de NVIDIA paso 3},basicstyle=\large]
	 blacklist nouveau
	 \end{lstlisting}
	 \item Tercero, una vez realizado esto tienes que ejecutar los siguientes comando para que los cambios hagan efecto, aunque si no lo hacen tendr�s que reiniciar el ordenador:
	 \begin{lstlisting}[frame=none,language=sh,caption={Instalaci�n \textit{driver} de NVIDIA paso 4},basicstyle=\large]
	 echo options nouveau modeset=0 | sudo tee -a /etc/modprobe.d/nouveau-kms.conf
	 update-initramfs -u
	 \end{lstlisting}
\end{itemize}

Ahora que se ha desactivado el controlador libre ``noveau'', procedemos a la instalaci�n del \textit{driver}, pero este debe ser sin interfaz gr�fica. Para trabajar sin interfaz gr�fica, primero, debemos pulsar la secuencia de teclas: CTRL+ALT+F1. Esto iniciar� un shell sin interfaz gr�fica.

Antes que nada debemos identificarnos ante el sistema, para identificarnos deberemos introducir nuestro nombre de usuario y la contrase�a, la contrase�a se te solicitar� una vez hayas introducido el nombre de usuario.

Una vez en el shell se procede a detener el demonio, controlador interno de Unix, que se encarga de mantener en ejecuci�n la interfaz gr�fica, para ello tenemos que ejecutar la siguiente orden en el shell:
\begin{lstlisting}[frame=none,language=sh,caption={Instalaci�n \textit{driver} de NVIDIA paso 5},basicstyle=\large]
sudo service gdm stop
\end{lstlisting}
Ahora se procede a descargar el \textit{driver} con la secuencia que antes hemos comentado sobre el \textit{script} de instalaci�n:
\begin{lstlisting}[frame=none,language=bash,caption={Instalaci�n \textit{driver} de NVIDIA paso 6},basicstyle=\large,label={lst:install}]
sudo apt-get install curl
cd ~/Downloads/
curl -O "http://developer.download.nvidia.com/compute/cuda/6_5/rel/installers/cuda_6.5.14_linux_64.run"
chmod +x cuda_6.5.14_linux_64.run
./cuda_6.5.14_linux_64.run --kernel-source-path=/usr/src/linux-headers-`uname -r`/
\end{lstlisting}
Una vez ejecutado esto, se iniciar� la instalaci�n y deber�s aceptar la licencia e instalar todo lo que se te propone, los controladores NVIDIA, CUDA y los ejemplos, todos deben ser instalados en sus directorios por defecto.

Cuando esto haya terminado tendr�s ya instalado los \textit{drivers} de NVIDIA, pero a�n queda hacer un par de pasos, lanzar el demonio de la interfaz de nuevo y reiniciar el sistema, hazlo con los siguientes comandos:
\begin{lstlisting}[frame=none,language=bash,caption={Instalaci�n \textit{driver} de NVIDIA paso 7},basicstyle=\large,label={lst:install}]
sudo service gdm start
sudo reboot
\end{lstlisting}

Con esto ya tendremos el controlador de NVIDIA y CUDA instalados, siendo CUDA necesario para la instalaci�n de las dependencias del servidor y las herramientas.
\section{Instalando Servidor y sus Dependencias}
Para instalar el servidor se ha usado en parte una peque�a gu�a\footnote{\url{https://github.com/BVLC/caffe/wiki/Ubuntu-14.04-VirtualBox-VM}}. Gracias a esta gu�a se instal� correctamente las dependencias, a la que podr�s acceder mirando la bibliograf�a\cite{Wiki}.

Para usar el \textit{script} de instalaci�n deber�s haber seguido los dos anteriores pasos (\ref{CUDA}) (\ref{MATLAB}). Ahora para iniciar el proceso de instalaci�n deber�s ejecutar el siguiente comando en el \textit{shell}:
\begin{lstlisting}[frame=none,language=bash,caption={Instalar Servidor},basicstyle=\large,label={lst:installS}]
sudo . ./install.sh
\end{lstlisting}

\textbf{IMPORTANTE:} Si te encuentras en una m�quina virtual, deber�s ir al \textit{script} de instalaci�n y modificarlo \textbf{antes de ejecutar la instalaci�n}. Deber�s cambiar la linea 47 de la siguiente manera:
\begin{lstlisting}[frame=none,language=bash,caption={Cambio para m�quinas virtuales},basicstyle=\large,label={lst:vm}]
#Esta es la l�nea original en el script:
cp ~/Proyecto-Fin-de-Grado/servidor/Install/Makefile.config ~/caffe

#Esto es lo que se debe poner:
cp ~/Proyecto-Fin-de-Grado/servidor/Install/MaquinaVirtual/Makefile.config ~/caffe
\end{lstlisting}

Si tienes alg�n problema con la instalaci�n o con el \textit{script}, se procede a explicar paso a paso qu� hace el programa de instalaci�n y qu� deber�as hacer t� para instalarlo sin el uso del \textit{script}. Todo esto debe ser siempre realizado tras haber hecho los anteriores dos pasos (\ref{CUDA}) (\ref{MATLAB}).

En primer lugar se va a un directorio por defecto para hacer la instalaci�n y seguidamente se instala la herramienta ``git'' para clonar los proyectos desde Github.
\begin{lstlisting}[frame=none,language=bash,caption={Explicaci�n Instalaci�n 1},basicstyle=\large,label={lst:exp1}]
cd ~
apt-get install git
\end{lstlisting}
Vemos que hemos hecho la acci�n de irnos a un directorio por defecto, este directorio sera la base de toda la instalaci�n y, por tanto, no se recomienda cambiarlo para que la configuraci�n del servidor sea m�s sencilla.

Ahora se procede a instalar los paquetes esenciales de Linux, adem�s de actualizar las cabeceras del \textit{kernel} a su �ltima versi�n para evitar errores futuros.
\begin{lstlisting}[frame=none,language=bash,caption={Explicaci�n Instalaci�n 2},basicstyle=\large,label={lst:exp2}]
apt-get install build-essential
apt-get install linux-headers-`uname -r`
\end{lstlisting}
En este paso clonaremos nuestro proyecto al directorio principal de instalaci�n.
\begin{lstlisting}[frame=none,language=bash,caption={Explicaci�n Instalaci�n 3},basicstyle=\large,label={lst:exp3}]
cd ~
git clone https://github.com/garfio1/Proyecto-Fin-de-Grado.git
\end{lstlisting}

En el siguiente paso se ha copiado la l�nea de instalaci�n de dependencias de la Wiki\cite{Wiki} de Github para instalar Caffe. Esto instalar� todas las dependencias necesarias para Caffe.
\begin{lstlisting}[frame=none,language=bash,caption={Explicaci�n Instalaci�n 4},basicstyle=\large,label={lst:exp4}]
apt-get install -y libprotobuf-dev libleveldb-dev libsnappy-dev libopencv-dev libboost-all-dev libhdf5-serial-dev protobuf-compiler gfortran libjpeg62 libfreeimage-dev libatlas-base-dev git python-dev python-pip libgoogle-glog-dev libbz2-dev libxml2-dev libxslt-dev libffi-dev libssl-dev libgflags-dev liblmdb-dev python-yaml
easy_install pillow
\end{lstlisting}

Ahora se procede a clonar Caffe\cite{jia2014caffe} en nuestro directorio principal de instalaci�n.
\begin{lstlisting}[frame=none,language=bash,caption={Explicaci�n Instalaci�n 5},basicstyle=\large,label={lst:exp5}]
cd ~
git clone https://github.com/BVLC/caffe.git
\end{lstlisting}

Ahora se instalan todos los requisitos para poder instalar PyCaffe, esto no deber�a ser necesario porque no usamos PyCaffe, pero la evoluci�n del proyecto NeuralTalk apunta a que acabar� usando esta librer�a de Caffe para Python, as� que se instala aqu� tambi�n para tener trabajo ya hecho.
\begin{lstlisting}[frame=none,language=bash,caption={Explicaci�n Instalaci�n 5},basicstyle=\large,label={lst:exp5}]
cd caffe
cat python/requirements.txt | xargs -L 1 sudo pip install
\end{lstlisting}

Ahora se procede a hacer unos enlaces con los nombres de las librer�as Python para que PyCaffe funcione de manera adecuada y correcta.

\begin{lstlisting}[frame=none,language=bash,caption={Explicaci�n Instalaci�n 6},basicstyle=\large,label={lst:exp6}]
sudo ln -s /usr/include/python2.7/ /usr/local/include/python2.7
sudo ln -s /usr/local/lib/python2.7/dist-packages/numpy/core/include/numpy/ /usr/local/include/python2.7/numpy
\end{lstlisting}

En el siguiente paso el programa de instalaci�n supone que has seguido todos los pasos hasta ahora como se han ido comentando y sustituye el Makefile.config de Caffe por el que viene en el proyecto. Puesto que si has instalado las cosas seg�n se han ido comentando, este archivo deber�a servirte para poder hacer los \textit{make} (instalaci�n a trav�s de un Makefile) de Caffe.
\begin{lstlisting}[frame=none,language=bash,caption={Explicaci�n Instalaci�n 7},basicstyle=\large,label={lst:exp7}]
rm ~/caffe/Makefile.config
cp ~/Proyecto-Fin-de-Grado/servidor/Install/Makefile.config ~/caffe
\end{lstlisting}

Finalmente se ejecuta la instalaci�n de Caffe a trav�s de la herramienta make, se instalan tanto MatCaffe como PyCaffe.
\begin{lstlisting}[frame=none,language=bash,caption={Explicaci�n Instalaci�n 8},basicstyle=\large,label={lst:exp8}]
make pycaffe
make matcaffe
make all
make test
\end{lstlisting}

En este punto nos falta el clonado del proyecto NeuralTalk en nuestra m�quina, por tanto, es lo que se procede a hacer en el pen�ltimo paso de la instalaci�n
\begin{lstlisting}[frame=none,language=bash,caption={Explicaci�n Instalaci�n 9},basicstyle=\large,label={lst:exp9}]
cd ~
git clone https://github.com/karpathy/neuraltalk.git
\end{lstlisting}

Finalmente nos resta �nicamente instalar las librer�as espec�ficas usadas en el servidor para pasar a la configuraci�n del proyecto para que funcione de manera adecuada.
\begin{lstlisting}[frame=none,language=bash,caption={Explicaci�n Instalaci�n 10},basicstyle=\large,label={lst:exp10}]
pip install beautifulsoup
pip install flask
\end{lstlisting}

Con esto tenemos instalado en nuestra m�quina todas las herramientas necesarias para el funcionamiento del servidor. Ahora hay que pasar a la configuraci�n  del mismo. Como se ha explicado en el apartado de Aspectos Relevantes (\ref{chap:AspectosRelevantes}), la configuraci�n no ha sido nada sencilla; pero se ha preparado para que ahora sea mucho m�s sencilla y r�pida.
\section{Configuraci�n del Servidor}
En este paso se proceder� a explicar c�mo se configura el servidor para que funcione, ahora resulta una tarea bastante sencilla porque se han preparado una serie de variables y ficheros para que esto sea as�.

En primer lugar, y lo m�s importante, es tener claro el directorio principal de la instalaci�n, pues todas las herramientas deber�an estar instaladas en �l. Si se ha seguido correctamente los pasos y se h mantenido el mismo directorio principal de instalaci�n, deber�as ser capaz de obtenerlo con la siguiente secuencia de comandos:
\begin{lstlisting}[frame=none,language=bash,caption={Configurando el servidor 1},basicstyle=\large,label={lst:conf1}]
cd ~
pwd
\end{lstlisting}
\figura{1}{imgs/Paso1.png}{Comprobando directorio principal de instalaci�n}{dirPrin}{}
El resultante de ejecutar los anteriores comandos es una cadena de caracteres que te comunican cu�l es el directorio principal de la instalaci�n, por ejemplo a mi me devuelve ``/home/bryan'', como se puede ver en la imagen \ref{dirPrin}.

Antes de continuar con la instalaci�n, debemos cambiar los permisos de las carpetas NeuralTalk y Proyecto-Fin-de-Grado para poder hacer la configuraci�n sobre ellas,  esto se hace de la siguiente manera:
lugar ejecutaremos el siguiente comando para empezar con la configuraci�n:
\begin{lstlisting}[frame=none,language=bash,caption={Configurando el servidor 2},basicstyle=\large,label={lst:conf2}]
chmod -R 777 ~/Proyecto-Fin-de-Grado
chmod -R 777 ~/neuraltalk
\end{lstlisting}

Una vez se ha detectado este directorio, podemos empezar con la configuraci�n del servidor. En primer lugar ejecutaremos el siguiente comando para empezar con la configuraci�n:
\begin{lstlisting}[frame=none,language=bash,caption={Configurando el servidor 3},basicstyle=\large,label={lst:conf3}]
cd ~/Proyecto-Fin-de-Grado/servidor
\end{lstlisting}
Con el anterior comando nos situamos en la carpeta en la que tendremos el primer documento a modificar. Tendremos que abrir con un editor de texto el archivo ``serv.py'' que nos encontramos en dicha carpeta. Podemos hacerlo con el siguiente comando si queremos, aunque el procesador de texto es a elecci�n del usuario:
\begin{lstlisting}[frame=none,language=bash,caption={Configurando el servidor 4},basicstyle=\large,label={lst:conf4}]
vi serv.py
\end{lstlisting}
Una vez tengamos abierto el archivo, deberemos localizar dos l�neas en concreto. Las l�neas a localizar con en las que se define la ruta del directorio principal de instalaci�n y en la que se define el usuario definido en la licencia de MATLAB. Podemos ver las l�neas a continuaci�n:
\begin{lstlisting}[frame=none,language=Python,caption={Configurando el servidor 5},basicstyle=\large,label={lst:conf5}]
#Deber�an estar en las l�neas 12 y 24 respectivamente pero si no lo est�n s�lo busca lo siguiente:
ROOT='/home/bryan'

USER='bryan'
\end{lstlisting}
Una vez hemos localizado las l�neas anteriores, en la constante ``ROOT'' deber�s poner el directorio que obtuviste en \ref{lst:conf1} o tu directorio principal de instalaci�n, el c�digo est� preparado para que, si has instalado las herramientas en el directorio correcto, no tengas que modificar m�s variables ni que estar picando el c�digo en busca de problemas; y en la constante ``USER'', deber�s poner el nombre de usuario que utilizaste al realizar el paso de Activaci�n de MATLAB(\ref{ActMATLAB}). Con estas modificaciones ya hemos configurado la mitad del servidor, s�lo nos queda ahora hacer las modificaciones pertinentes para que la herramienta NeuralTalk funcione en nuestra m�quina.

Para empezar, junto con el proyecto se facilita una copia del archivo ``extract\_ features.m'' de NeuralTalk, s�lo que este archivo est� modificado para que la configuraci�n se muy sencilla. Lo que haremos ser� borrar el archivo que viene por defecto junto con la herramienta NeuralTalk y se copiar� el que viene adjunto con el proyecto, lo haremos de la siguiente manera:
\begin{lstlisting}[frame=none,language=bash,caption={Configurando el servidor 6},basicstyle=\large,label={lst:conf6}]
rm ~/neuraltalk/matlab_features_reference/extract_features.m
cp ~/Proyecto-Fin-de-Grado/servidor/Install/extract_features.m ~/neuraltalk/matlab_features_reference/
\end{lstlisting}

Ahora debemos dirigirnos a la carpeta en la que acabamos de copiar el fichero y abrirlo para proceder a modificarlo, esto se puede hacer v�a comandos de shell o puede el usuario abrirlo con cualquier procesador de texto que el prefiera:
\begin{lstlisting}[frame=none,language=bash,caption={Configurando el servidor 7},basicstyle=\large,label={lst:conf7}]
cd ~/neuraltalk/matlab_features_reference/
vi extract_features.m
\end{lstlisting}

Una vez tenemos el fichero abierto para su modificaci�n debemos buscar una l�nea si estamos en una m�quina anfitriona, dos si estamos en una m�quina virtual:
\begin{lstlisting}[frame=none,language=Matlab,caption={Configurando el servidor 8},basicstyle=\large,label={lst:conf8}]
%Si estamos en m�quina virtual buscamos tambi�n la siguiente l�nea, t�picamente est� en la l�nea 3:
use_gpu = 1;

%Deber�a estar en la l�nea 5
root='/home/bryan';
\end{lstlisting}

Si nos encontramos en una m�quina virtual, la variable ``use\_ gpu''  deber�amos ponerla con el valor  0. Mientras que lo siguiente se hace tanto para m�quina virtual como para m�quina anfitriona, se cambiar� la variable ``root'' por el directorio que se obtuvo en \ref{lst:conf1}.

Con esto ya tenemos hecho las modificaciones pertinentes para que el servidor funcione correctamente, s�lo nos queda descargar un fichero necesario para la ejecuci�n del \textit{script} de MATLAB. El fichero en cuesti�n ser� descargado de la siguiente manera:
\begin{lstlisting}[frame=none,language=Matlab,caption={Configurando el servidor 9},basicstyle=\large,label={lst:conf9}]
cd ~/Proyecto-Fin-de-Grado/servidor/uploads
wget "http://www.robots.ox.ac.uk/~vgg/software/very_deep/caffe/VGG_ILSVRC_16_layers.caffemodel"
\end{lstlisting}

Ahora ya podemos ejecutar el servidor, para ello podremos ejecutar los siguientes comandos:
\begin{lstlisting}[frame=none,language=Matlab,caption={Ejecutando el servidor},basicstyle=\large,label={lst:ex}]
cd ~/Proyecto-Fin-de-Grado/servidor
sudo python serv.py
\end{lstlisting}
%
%\portadasAuxiliares{Anexo V - Manual del usuario}
%\chapter{Manual del usuario}
%\section{Introducci�n}
En este anexo se presentar� el manual de usuario, d�nde se explicar� c�mo instalar y usar la aplicaci�n.
\subsection{Instalar Aplicaci�n}
Si queremos extraer la aplicaci�n del proyecto, deberemos dirigirnos al directorio ``/Proyecto-Fin-de-Grado/Imagen/app'', en �l encontraremos un archivo llamado ``app-release.apk'', este archivo se guardar� en el dispositivo en el que se vaya a instalar la aplicaci�n.
\begin{figure}[]
    \begin{center}%
      \Ovalbox{%
        \begin{minipage}{\anchoFloat}%
          \begin{center}%
            \begin{tabular}{ l r}
 				\includegraphics[width=0.30\textwidth]{imgs/us1.png} & 
 				\includegraphics[width=0.30\textwidth]{imgs/us2.png} \\
			\end{tabular}
            \caption{Instalando aplicaci�n paso 1.}%
            \label{app1}%
          \end{center}%
        \end{minipage}
      }%
    \end{center}%
  \end{figure}%
  \begin{figure}[]
    \begin{center}%
      \Ovalbox{%
        \begin{minipage}{\anchoFloat}%
          \begin{center}%
            \begin{tabular}{ l r}
 				\includegraphics[width=0.30\textwidth]{imgs/us3.png} & 
 				\includegraphics[width=0.30\textwidth]{imgs/us4.png} \\
			\end{tabular}
            \caption{Instalando aplicaci�n paso 2.}%
            \label{app2}%
          \end{center}%
        \end{minipage}
      }%
    \end{center}%
  \end{figure}%
  \figura{0.45}{imgs/us5.png}{Instalando aplicaci�n paso 3}{app3}{}
  Hay que destacar que la instalaci�n de la aplicaci�n se debe hacer de forma manual, a�n no se ha planificado ni modelado una manera de que una persona invidente pueda hacerlo, por tanto esta deber� tener alguien que le ayude a instalarla.
  
  Para empezar la instalaci�n tenemos que buscar en los archivos del dispositivo el fichero de instalaci�n con el nombre ``app-release.apk'', una vez encontrado tocamos la pantalla sobre este y se abrir� una ventana de instalaci�n. En la ventana de instalaci�n se nos muestra los permisos que requiere la aplicaci�n y nos da la opci�n de instalar, presionamos al bot�n de instalar para proceder con la instalaci�n. Estos pasos est�n representados en la imagen \ref{app1}.
  
  Despu�s de haber presionado el bot�n de instalar, se nos muestra una ventana que nos informa de que la aplicaci�n se est� instalando. Seguidamente, una vez se ha instalado la aplicaci�n, se mostrar� una ventana inform�ndonos de que la aplicaci�n se ha instalado correctamente y nos dar� la opci�n de abrirla.Esto esta representado en la imagen \ref{app2}.
  
  Si hemos abierto la aplicaci�n se nos mostrar� la pantalla inicial (\ref{app3}).

\section{Uso de la aplicaci�n}
Cuando se ha iniciado la aplicaci�n y se nos muestra la pantalla inicial (\ref{app3}), el usuario deber� apuntar con el m�vil en la direcci�n que se quiera tomar la foto y tocar la pantalla. Entonces la aplicaci�n se encargar� de mandar la foto y recibir la predicci�n.

Si el usuario no sabe qu� tiene que hacer cuando abre la aplicaci�n, pasado un tiempo, se le lee un mensaje en voz alta con la librer�a \textit{TText to Speech} dici�ndole qu� tiene que hacer.

Tras la ejecuci�n de la petici�n al servidor y el procesado de la imagen, la aplicaci�n leer� en voz alta la predicci�n. Y aqu� termina la ejecuci�n de la aplicaci�n, si se quiere hacer otra predicci�n, se debe reiniciar la aplicaci�n; pues, de momento, la aplicaci�n no tiene una ejecuci�n circular.


%
% AP�NDICES
%\backmatter
%\appendix

% A�adir entrada en el �ndice: Ap�ndices
%\addcontentsline{toc}{chapter}{Ap�ndices}
%
%\portadasAuxiliares{Ap�ndice A - Gu�a r�pida de Version One}
%\chapter{Gu�a r�pida de Version One}
%\input{apendiceA}
%
%\portadasAuxiliares{Ap�ndice B - Licencia GNU GPL}
%\chapter{Licencia GNU GPL}
%\input{apendiceB}

%
%Bibliograf�a
\normalem
\bibliografia{referencias}


% Otras referencias
%\bibliografiaOtras{otrasreferencias}

\end{document}
